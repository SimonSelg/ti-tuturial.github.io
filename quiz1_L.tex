
\documentclass{scrartcl}[11pt]
\usepackage{german}
\usepackage[utf8]{inputenc}
\usepackage[german]{babel}
\usepackage{amssymb}
\usepackage{amsmath}
\usepackage{graphicx}
\usepackage{fancyhdr}
\usepackage{lastpage}
\usepackage{braket}
\usepackage{listings}
\usepackage{tikz}
\usetikzlibrary{arrows,automata,positioning}

\setlength{\parskip}{\medskipamount}
\setlength{\parindent}{1pt}


%%%%%%%%%%%%%%%%%%%%%%%%
% Kopf- und Fusszeilen %
%%%%%%%%%%%%%%%%%%%%%%%%
\pagestyle{fancy}
\lhead{TI-Tutorial}
\chead{Quiz 1}
\rhead{20 Minuten}
\lfoot{}
\cfoot{}
\rfoot{}

\begin{document}

%_______________________________________________________________________________
%_______________________________________________________________________________
\section*{Lösung 1}
\emph{Korrekturhinweise (streng):}
\begin{itemize}
  \item Aufteilung in $L_i^M$ fehlt oder garnicht verstanden \emph{[-10 Pkt]}
  \item OFF-Menge benutzt \emph{[-8 Pkt]}
  \item $Prim_i$ fehlt \emph{[-3 Pkt]}
  \item $L_0$ fehlt \emph{[-2.5 Pkt]}
  \item Partitionierung innerhalb $L_i^M$'s fehlt \emph{[-2.5 Pkt]}
  \item Schleifendurchlauf 4 nicht angegeben \emph{[-2 Pkt]}
  \item Primimplikanten zu falschem Zeitpunkt hinzugefügt \emph{[-2 Pkt]}
  \item Einzeln fehlende $L_i^M$, einmalig \emph{[-1.5 Pkt]}
  \item Minimalpolynom nicht angegeben \emph{[-1 Pkt]}
  \item $Prim_i$ fortlaufend erweitert \emph{[-1 Pkt]}
  \item Pro Fehler bei OFF-Mengen überführung \emph{[-0.5 Pkt]}
  \item Pro falschem/fehlenden Implikanten, auch für Folgefehler innerhalb des Algorithmus \emph{[-0.5 Pkt]}
  \item Abbruch Bedingung fehlt \emph{[-0.5 Pkt]}
  \item Sonstige leichte Notationsfehler \emph{[-0.5 Pkt]}
\end{itemize}

\subsection*{Initialisierung}

\begin{tabular}{cc}
  \begin{tabular}{c}
    $L_0^{\{x_1, x_2, x_3, x_4\}}$ \\ \hline \hline
  \end{tabular} & $Prim_f = \emptyset$ \\
  \begin{tabular}{c}
0000 \\ \hline
0001 \\
0100 \\
1000 \\ \hline
0011 \\
0101 \\
1001 \\
1010 \\
1100 \\ \hline
0111 \\
1101 \\ \hline
1110 \\

  \end{tabular}
\end{tabular}

%-------------------------------------------------------------------------------
\subsection*{1. Schleifendurchlauf}

\begin{tabular}{cccc}

  \begin{tabular}{c}
    $L_1^{\{x_1, x_2, x_3\}}$ \\ \hline \hline
    000- \\  \hline
    010- \\
    100- \\ \hline
    110- \\
  \end{tabular} &


  \begin{tabular}{c}
    $L_1^{\{x_1, x_2, x_4\}}$ \\ \hline \hline
    00-1 \\
    10-0 \\ \hline
    01-1 \\
    11-0 \\

  \end{tabular} &


  \begin{tabular}{c}
    \\
    \\
    $L_1^{\{x_1, x_3, x_4\}}$ \\ \hline \hline
    0-00 \\ \hline
    0-01 \\
    1-00 \\ \hline
    0-11 \\
    1-01 \\
    1-10 \\
  \end{tabular} &


  \begin{tabular}{c}
    $L_1^{\{x_2, x_3, x_4\}}$ \\ \hline \hline
    -000 \\ \hline
    -001 \\
    -100 \\ \hline
    -101 \\

  \end{tabular} \\

  $Prim_f = \emptyset$

\end{tabular}


%-------------------------------------------------------------------------------
\subsection*{2. Schleifendurchlauf}


\begin{tabular}{cccccc}

  \begin{tabular}{c}
    $L_2^{\{x_1, x_2\}}$ \\ \hline \hline
    \\
    \\
  \end{tabular} &


  \begin{tabular}{c}
    $L_2^{\{x_1, x_3\}}$ \\ \hline \hline
    0-0- \\ \hline
    1-0- \\
  \end{tabular} &


  \begin{tabular}{c}
    $L_2^{\{x_1, x_4\}}$ \\ \hline \hline
    1$--$0 \\
    0$--$1 \\
  \end{tabular} &

  \begin{tabular}{c}
    $L_2^{\{x_2, x_3\}}$ \\ \hline \hline
    -00- \\ \hline
    -10- \\
  \end{tabular} &

  \begin{tabular}{c}
    $L_2^{\{x_2, x_4\}}$ \\ \hline \hline
    \\
    \\
  \end{tabular} &

  \begin{tabular}{c}
    $L_2^{\{x_3, x_4\}}$ \\ \hline \hline
    $--$00 \\ \hline
    $--$01 \\
  \end{tabular} \\

  $Prim_f = \emptyset$

\end{tabular}


%-------------------------------------------------------------------------------
\subsection*{3. Schleifendurchlauf}

\begin{tabular}{cccc}

  \begin{tabular}{c}
    $L_3^{\{x_1\}}$ \\ \hline \hline
  \end{tabular} &

  \begin{tabular}{c}
    $L_3^{\{x_2\}}$ \\ \hline \hline
  \end{tabular} &

  \begin{tabular}{c}
    \\
    $L_3^{\{x_3\}}$ \\ \hline \hline
    $--$0-
  \end{tabular} &

  \begin{tabular}{c}
    $L_3^{\{x_4\}}$ \\ \hline \hline
  \end{tabular} \\
\end{tabular}

  $Prim_f = \{ 0--1, 1--0 \}$


%-------------------------------------------------------------------------------
\subsection*{4. Schleifendurchlauf}


\begin{tabular}{c}
  L_4^{\{ \}} \\ \hline \hline
\end{tabular}
\\

$Prim_f = \{ 0--1, 1--0, --0- \}$

%-------------------------------------------------------------------------------
\subsection*{Abbruch}

\begin{align*}
 &\bigcup_M L_4^M(f) = \emptyset \\
\Rightarrow & \mbox{ Abbruch der Schleife und } return \ Prim(f)
\end{align*}

\pagebreak

%_______________________________________________________________________________
%_______________________________________________________________________________
\section*{Lösung 2}

\emph{Korrekturhinweise (streng):}
\begin{itemize}
  \item[a)]
  \begin{itemize}
    \item Nachkommastellen nicht mitangegeben \emph{[-1.5 Pkt.]}
    \item $n$ Vorkomma stellen benutzt \emph{[-1 Pkt]}
    \item Einerkomplement hingeschrieben \emph{[-1.5 Pkt.]}
    \item Klammerung nicht eindeutig \emph{[-0.5 Pkt]}
    \item Kleine Fehler \emph{[-1 Pkt.]}
  \end{itemize}
  \item[b)] Jeweils \emph{[1.5 Pkt]}, Richtig/Falsch, keine Teilpunkte
  \item[c)]
  \begin{itemize}
    \item $d$ statt $a$ benutzt \emph{[-1 Pkt.]}
    \item Klammerung nicht eindeutig, jeweils \emph{[-0.5 Pkt]} maximal \emph{[-1 Pkt.]}
    \item Für jeden fehlenden Schritt (vgl. Musterlösung), der nicht klar herausgestellt wurde \emph{[-1 Pkt.]}
    \item Für $n+1$ Vorkomma stellen gezeigt \emph{[-1 Pkt.]}
    \item Lösungen mit falscher Definitionen von Zweierkomplement können maximal $2.5$ Punkte erhalten.
    \item Nicht $k=0$ gesetzt \emph{[-1 Pkt]}
    \item Die $-1$ nicht heraus gezogen \emph{[-1 Pkt.]}
  \end{itemize}
\end{itemize}


\emph{Lösungen}
\begin{itemize}
  \item[a)] $$[d]_2 := \left(\sum_{i=-k}^{n} d_i \cdot 2^i\right) - d_{n+1} \cdot 2^{n+1} $$
  \item[b)] \begin{align*}
    [0101.10]_2 &= 5.5_{dez} \\
    [1001.01]_2 &= -6.75_{dez}
  \end{align*}
  \item[c)]

  \begin{align*}
    [\bar{a}]_2 &= \left(\sum_{i=0}^{n-1} (1-a_i) \cdot 2^i\right) - (1-a_n) \cdot 2^{n}\\
    &= \left(\sum_{i=0}^{n-1}  2^i - a_i \cdot 2^i \right) - (2^n - a_n \cdot 2^n)\\
    &= \left(\sum_{i=0}^{n-1}  2^i\right) - \left(\sum_{i=0}^{n-1}a_i \cdot 2^i \right) - (2^n - a_n \cdot 2^n)\\
    &\overset{GS}{=} (2^n - a_n \cdot 2^n) - \left(\sum_{i=0}^{n-1}a_i \cdot 2^i \right) - (2^n - a_n \cdot 2^n)\\
    &= - \left(\sum_{i=0}^{n-1}a_i \cdot 2^i \right) \\
    &\overset{Def}{=} -[a]_2
  \end{align*}

\end{itemize}


\end{document}
