\documentclass{scrartcl}
\usepackage{geometry}
\usepackage[utf8]{inputenc}
\usepackage[german]{babel}

% zusätzliche mathematische Symbole, AMS=American Mathematical Society 
\usepackage{amssymb}
\usepackage{amsmath}
\usepackage{pgf,tikz}
%\usetikzlibrary{positioning,shapes,arrows,decorations.pathreplacing,backgrounds,snakes}

% fürs Einbinden von Graphiken
\usepackage{graphicx}

% für Namen etc. in Kopf- oder Fußzeile
\usepackage{fancyhdr}

% erlaubt benutzerdefinierte Kopfzeilen 
\pagestyle{fancy}


% für COde
\usepackage{listings}
\usepackage{color}

\definecolor{mygreen}{rgb}{0,0.6,0}
\definecolor{mygray}{rgb}{0.5,0.5,0.5}
\definecolor{mymauve}{rgb}{0.58,0,0.82}
\definecolor{mylightgray}{rgb}{0.98,0.98,0.98}
\definecolor{mydarkgreen}{rgb}{0.0,0.5,0.0}
\definecolor{mymagenta}{rgb}{0.5,0.0,0.5}
\definecolor{mybrown}{rgb}{0.9,0.5,0.2}

\lstset{ %
  backgroundcolor=\color{mylightgray},   % choose the background color; you must add \usepackage{color} or \usepackage{xcolor}
  basicstyle=\footnotesize,        % the size of the fonts that are used for the code
  breakatwhitespace=false,         % sets if automatic breaks should only happen at whitespace
  breaklines=true,                 % sets automatic line breaking
  captionpos=b,                    % sets the caption-position to bottom
  commentstyle=\color{red},        % comment style
  deletekeywords={},            % if you want to delete keywords from the given language
  escapeinside={\%*}{*)},          % if you want to add LaTeX within your code
  extendedchars=true,              % lets you use non-ASCII characters; for 8-bits encodings only, does not work with UTF-8
  frame=single,	                   % adds a frame around the code
  keepspaces=true,                 % keeps spaces in text, useful for keeping indentation of code (possibly needs columns=flexible)
  keywordstyle=\color{blue},       % keyword style
  language=Python,                 % the language of the code
  otherkeywords={*,...},           % if you want to add more keywords to the set
  %numbers=left,                    % where to put the line-numbers; possible values are (none, left, right)
  numbersep=5pt,                   % how far the line-numbers are from the code
  numberstyle=\tiny\color{mygray}, % the style that is used for the line-numbers
  rulecolor=\color{black},         % if not set, the frame-color may be changed on line-breaks within not-black text (e.g. comments (green here))
  showspaces=false,                % show spaces everywhere adding particular underscores; it overrides 'showstringspaces'
  showstringspaces=false,          % underline spaces within strings only
  showtabs=false,                  % show tabs within strings adding particular underscores
  stepnumber=2,                    % the step between two line-numbers. If it's 1, each line will be numbered
  stringstyle=\color{mydarkgreen},     % string literal style
  tabsize=2,	                   % sets default tabsize to 2 spaces
  title=\lstname,                   % show the filename of files included with \lstinputlisting; also try caption instead of title 
   classoffset=1,
   morekeywords={def},keywordstyle=\color{red},
   classoffset=2,
   morekeywords={print,True,False},keywordstyle=\color{mymagenta},
   classoffset=3,
   emph={return,yield},
   emphstyle=\color{mybrown},
   classoffset=0
   }


%emph1={def},
%emph1style=\color{blue}\textbf,
%emph2={yield,return},          % Custom highlighting
%emph2style=\color{red},
%emph3={print,True,False},          % Custom highlighting
%emph3style=\color{mymagenta},
%} 






\usepackage{array}
\usepackage{pgf,tikz}
\usepackage{ amssymb }
% \usetikzlibrary{arrows,automata}
% Definition der Kopfzeile
\lhead{
Lösung zur Klausur \#1982
}
\chead{}
\rhead{\today{}}
\lfoot{}
\cfoot{Seite \thepage}
\rfoot{} 

\begin{document}

\section*{Lösung zur Klausur \#1982}

Es wird keine Garantie für die Richtigkeit gegeben.
Diese Lösung ist von Studenten angefertigt.
\subsection*{Aufgabe 1}
\subsubsection*{a)}
\begin{lstlisting}
>>> 'I %s %sam' % ('spam'[2:], 'spam'[:2])
'I am spam'
\end{lstlisting}

\subsubsection*{b)}
\begin{lstlisting}
>>> a=((i, j) for i in range(1, 4) for j in range(1, 7, 2)
...           if i**2 < j*i)
>>> tuple(a)
((1, 3), (1, 5), (2, 3), (2, 5), (3, 5))
\end{lstlisting}

\subsubsection*{c)}
\begin{lstlisting}
>>> import re
>>> re.findall(r'([a-z]+?)\w*', "Ham, spam, and, eggs")
['a', 's', 'a', 'e']
\end{lstlisting}

\subsubsection*{d)}
\begin{lstlisting}
>>> from operator import itemgetter
>>> a = [3, 2, 1]
>>> b = ("a", "C", "b")
>>> sorted(zip(a, b), key=itemgetter(1), reverse=True)
[(1, 'b'), (3, 'a'), (2, 'C')]
\end{lstlisting}

\subsubsection*{e)}
\begin{lstlisting}
>>> a, b = (3, 2, 1, 0), (2, 1 , 3, 4, 5)
>>> list(map(max, a, b))
[3, 2, 3, 4]
\end{lstlisting}

\subsubsection*{f)}
\begin{lstlisting}
>>> f = lambda y: lambda x: int(x - y/2)
>>> f(-56)(14)
42
\end{lstlisting}

\subsection*{Aufgabe 2}
\subsubsection*{a)}
\begin{itemize}
\item statische Finitheit
\item dynamische Finitheit
\item Effektivität
\item Präzise
\item Terminiert
\end{itemize}

\subsubsection*{b)}
imperative Programmierung: Man beschreibt,
 wie
 etwas
erreicht werden soll

deklarative Programmierung: Man beschreibt,
 was
erreicht werden soll.

\subsubsection*{c)}
\begin{tabular}{c|c}
Imperativ & Deklarativ \\
\hline
prozedurale Programierung & funktionale Programmierung\\
OOP & 
\end{tabular}

\subsubsection*{d)}

\begin{itemize}
 \item 1936 Turing-Maschine
 \item 1941 Erfindung des Z3
 \item 1948 Erster Speicherprogrammierbarer Computer
 \item 1971 Erster Mikroprozessor
 \item 1989 Beginn der Entwicklung des WWW
\end{itemize}

\subsection*{Aufgabe 3}
\subsubsection*{a)}
\begin{lstlisting}
def sieve(n):
    siv=set(range(2,n+1))
    for i in range(2, n+1):
        if i in siv:
            for x in range(i**2, n+1):
                if x % i == 0:
                    siv.discard(x)
    return siv

sieve(20)
\end{lstlisting}

\subsubsection*{b)}
\begin{lstlisting}
def isprime(n):
    if n == 0 or n==1:
        return False
    if n in sieve(n):
        return True
    else:
        return False

isprime(20)
\end{lstlisting}

\subsection*{Aufgabe 4}
\subsubsection*{a)}
\begin{lstlisting}
def reachable(sender, receiver, confidants):
    marked = dict((x, False) for x in confidants)
    marked[sender] = True
    L = [sender]
    while L:
        print(marked, "\n" , L, "\n")
        cur = L.pop()
        if cur == receiver:
            return True
        for f in confidants[cur]:
            if not marked[f]:
                marked[f] = True
                L.append(f)
    return False

cnfd = {1: [2, 3], 2: [5, 4], 3: [3], 4: [6, 3], 5: [6], 6:[5]}
reachable(1, 6, cnfd)
\end{lstlisting}
Ausgabe:
\begin{lstlisting}
{1: True, 2: False, 3: False, 4: False, 5: False, 6: False} 
 [1] 

{1: True, 2: True, 3: True, 4: False, 5: False, 6: False} 
 [2, 3] 

{1: True, 2: True, 3: True, 4: False, 5: False, 6: False} 
 [2] 

{1: True, 2: True, 3: True, 4: True, 5: True, 6: False} 
 [5, 4] 

{1: True, 2: True, 3: True, 4: True, 5: True, 6: True} 
 [5, 6] 
\end{lstlisting}

\verb_True_

\subsubsection*{b)}
\(O(n^2)\)

\subsection*{Aufgabe 5}
\subsubsection*{a)}
\begin{lstlisting}
def fractions():
    zaehler, nenner, direction = 1, 1, "up"
    while True:
   		yield zaehler, nenner
        if zaehler == 1:
            if direction == "up":
                nenner += 1
                direction = "down"
            else:
                zaehler += 1
                nenner -= 1
        elif nenner == 1:
            if direction == "down":
                zaehler += 1
                direction = "up"
            else:
                zaehler -= 1
                nenner += 1
        elif direction == "up":
            zaehler -= 1
            nenner += 1
        else:
            zaehler += 1
            nenner -= 1
\end{lstlisting}

\subsubsection*{b)}

\begin{lstlisting}
def print_fractions(n):
    f = fractions()
    for i in range(n):
        print(next(f))
\end{lstlisting}

\subsection*{Aufgabe 6}

\begin{lstlisting}
user = 'Alice'
def decorator(f):
    user = 'Bob'
    def wrapper(*args, **kwargs):
        global user
        res = '%s %s' % (user, f(*args, **kwargs))
        return res
    return wrapper

@decorator
def func1(str):
    return '%s im Hoersaal.' % str

def func2(str):
    return '%s %s ein Buch.' % (user, str)

print(func1('schreibt'))
print(func2('liest'))
\end{lstlisting}    

\subsubsection*{a)}

\begin{verbatim}
Bob schreibt im Hoersaal.
Alice liest ein Buch.
\end{verbatim}

\subsubsection*{b)}

\begin{verbatim}
Alice schreibt im Hoersaal.
Alice liest ein Buch.
\end{verbatim}

\subsection*{Aufgabe 7}

\subsubsection*{a)}

\begin{lstlisting}[escapechar=@]
class Marmelade:
    step = 5
    def __init__(@\colorbox{yellow}{self,}@ start, limit):
        self.run = start
        self.limit = limit

    def __iter__(self):
        return self

    def __next__(self):
        self.run += @\colorbox{yellow}{self.}@step
        if self.run > self.limit:
            raise StopIteration
        return @\colorbox{yellow}{self.}@run

jam = Marmelade(32, 45)
for i in jam:
    print(i, end=', ')

for i in jam:
    print(2*i, end=', ')

print("Empty!")

\end{lstlisting}

\subsubsection*{b)}

\begin{verbatim}
37, 42, Empty
\end{verbatim}

\subsubsection*{c)}

Iteratorprotokoll

\subsection*{Aufgabe 8}

\subsubsection*{a)}
\verb_afunc_
\[ \prod _{i=k} ^n f(i)\]

\verb_bfunc_
\[ \prod _{i=1} ^n 2 \cdot n\]

\subsubsection*{b)}

\begin{lstlisting}[escapechar=@]
def cfunc(f, k, n):
    if @\colorbox{yellow}{k=n+1}@:
        return 1
    else:
        return @\colorbox{yellow}{f(n)}@*cfunc@\colorbox{yellow}{f,k,n-1}@
\end{lstlisting}

\end{document}