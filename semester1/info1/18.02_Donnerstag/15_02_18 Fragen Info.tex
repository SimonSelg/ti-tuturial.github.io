\documentclass{scrartcl}[10pt]
\usepackage{german}
\usepackage[latin1]{inputenc}
\usepackage[german]{babel}
\usepackage{dsfont}
% zus\"atzliche mathematische Symbole, AMS=American Mathematical Society 
\usepackage{amssymb}
\usepackage{amsmath}
\usepackage{ulem}
\usepackage{colortbl}
\usepackage{xcolor}

% f\"urs Einbinden von Graphiken
\usepackage{graphicx}

% f\"ur Namen etc. in Kopf\textendash{} oder Fu\ss zeile
\usepackage{fancyhdr}
\usepackage{float}

% erlaubt benutzerdefinierte Kopfzeilen 
\pagestyle{fancy}

% Definition der Kopfzeile
\lhead{
\begin{tabular}{ll}
Pasquale Malm & 2727102 \\
Panajiotis Christoforidis & 4128421
\end{tabular}
}
\chead{}
\rhead{\today{}}
\lfoot{}
\cfoot{Seite \thepage}
\rfoot{} 

\begin{document}

\section*{Frage 1}
Welche ethischen Leitregeln gibt es f\"r Informatiker? Nenne 3\\
\textbf{GI} (Gesellschaft für Informatik)\\
\textbf{ACM} (Association for Computing Machinery)\\
\textbf{VDI} (Verbund deutscher Informatiker)

\section*{Frage 2}
Nenne die drei Voraussetzungen zur Entwicklung des Computers\\
Techniken zur Atomatisierung\\
Mathematische Grundlagen\\
Basistechnologien\\

\section*{Frage 3}
Verbinde die Ereignisse mit den Jahreszahlen?\\
1941 - Z3 der erste funtionierende Computer\\
1936 - Church-Turing\\
1843 - Ada Lovelace... schreibt das erste Programm\\
1673 - Die Leibnizmaschine\\

\section*{Frage 4}
Erkl\"are diese vier Konzepte: Input/Output, Algorithmus, Programm und Berechnungsprozess
\begin{itemize}

\item[Input/Output:] Eingabe an einen Prozess, die er verwenden kann. Output ist die Ausgabe des Programm.

\item[Algorithmus:] Eine Liste an Vorschriften zur L\"osung eines Problems.

\item[Programm:] Implementierung eines Algorithmus.

\item[Berechnungsprozesse:] Eine Instanz des Programmes.

\end{itemize}

\section*{Frage 5}
Nennen Sie w\"unschenswerte Eigenschaften f\"ur Algorithmen
\begin{itemize}

\item[Determinismus:] Die Folgeschritte sind immer eindeutig festgelegt. 

\item[Terminiertheit:] F\"ur einen gegebenen Input ist der Output immer der gleiche.

\item[Generalit\"at:] Der Algorithmus liefert eine Vorschrift f\"ur eine ganze Klasse an Problemen.

\end{itemize}
\end{document}