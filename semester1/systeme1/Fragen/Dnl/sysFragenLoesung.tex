\documentclass[10pt,a4paper]{article}
\usepackage[utf8]{inputenc}
\usepackage[german]{babel}
\usepackage{amsmath}
\usepackage{amsfonts}
\usepackage{amssymb}
\author{Daniel}
\title{Systeme Fragen}
\begin{document}
\maketitle

\subsection*{Kapitel 2}

\begin{itemize}
\item  Welche Aufgaben hat ein Betribssystem? (allgemein)
\item[$\rightarrow$] Abstraktion der Hardware und Verwaltung der Systemressourcen
\item[$\rightarrow$] Geordnete und kontollierte Zuteilung der Betriebsmittel an konkuriende Prozesse / Benutzer
\item Nenne drei Aten der Prozessverwaltung
\item[$\rightarrow$] Batch - Verwaltung/Stapelverarbeitung
\item[$\rightarrow$] Transaktionsverfahren/Dialogverarbeitung
\item[$\rightarrow$] Timesharing
\end{itemize}

\subsection*{Kapitel 3}

\begin{itemize}
\item Nenne drei Dateiattribute:
\item[$\rightarrow$] Entstehungszeitpunkt
\item[$\rightarrow$] Dateigröße
\item[$\rightarrow$] Zugriffsrechte
\item Nenne drei Zugriffsarten
\item[$\rightarrow$] Lesend
\item[$\rightarrow$] Schreibend
\item[$\rightarrow$] Ausführung
\item Nenne drei Zugriffsgruppen
\item[$\rightarrow$] Dateieigentümer
\item[$\rightarrow$] Benutzergruppe des Eigentümers
\item[$\rightarrow$] Alle anderen Benutzer
\item Was bedeutet folgende Ausgabe von \verb_ls -l_: \verb_drwxrw-r-- ute women 42345 Oct 12 15:16 otto_
\item[$\rightarrow$] ...
\item \verb_rwsr-x---_
\item \verb_rwsr-xr-x_
\item \verb_rwxrws---_
\item \verb_drwxrwxrwt_
\item \verb_-rw-r--r--+_
\item Welche Arten von Links kennst du?
\item[$\rightarrow$] Symbolische Links/Softlinks und Hardlinks
\item Es werden beide Arten von Links auf die Datei\\ \verb_-rwxrwxrwx 231 meier users Oct 12 12:35 dat.txt_ erstellt. Wie sieht jetzt die Ausgabe von \verb_ls -l_ aus?
\item[$\rightarrow$] \verb_lrwxrwxrwx 267 meier users Oct 12 16:46 sym -> /home/mueller/dokumente/dat.txt_
\item[$\rightarrow$] \verb_-rwxrwxrwx 231 meier users Oct 13 14:57 hard_
\item Mit welchen Befehlen kann man die Links erstellen?
\item[$\rightarrow$] \verb_ln -s quelle Ziel_ bzw. \verb_ln quelle ziel_
\item Vorteile/Nachteile?
\item Operationen auf Dateien?
\item[$\rightarrow$] Create, Delete, Open, Close, Read, Write, Append, Seek,
Get Attributes, Set Attributes, Rename
\item Operationen auf Verzeichnisse?
\item[$\rightarrow$] Create, Delete, Opendir, Closedir, Readdir, get Attributes, Set Attributes, Rename
\item Arten von Dateizugriff?
\item[$\rightarrow$] Wahlfrei, sequenziell
\item Nenne drei Alternativen zur Realisierung von Speicherung von Dateien
\item[$\rightarrow$] Zusammenhängende Belegung
\item[$\rightarrow$] Verkette Listen
\item[$\rightarrow$] Inodes
\item Vorteile/Nachteile/Bsp. für zusammenhängende Belegung
\item[$\rightarrow$] Vorteile: Lesegeschwindigkeit
\item[$\rightarrow$] Nachteile: Externe Fragmentierung
\item[$\rightarrow$] Beispiele: CD/DVD/Blu-raj/CD-ROM/HD DVD o.ä.
\item Vorteile/Nachteile für verkette Listen
\item[$\rightarrow$] Vorteile:keine externe Fragmentierung, Dateien belibiger Größe können gesp. werden
\item[$\rightarrow$] Nachteile: langsamer wahlfreier zugriff
\item Grundaufbau FAT
\item[$\rightarrow$] Verkettete Liste mit Information über Verkettung im Hauptspeicher. FAT im Hauptspeicher
\item Vorteile/Nachteile für FAT
\item[$\rightarrow$] Vorteile: schnellerer wahlfreier Zugriff, da Kette von Verweisen verfolgt erden muss
\item[$\rightarrow$] Nachteile: Jeder Block hat Zeiger in FAT, max. partitionsgröße
\item Wie funktionieren I-Nodes? Was ist in ihnen gespeichert?
\item[$\rightarrow$]In den I-Nodes sind alle Plattenblockadressen und die Metadaten gespeichert. I-Node nur von offenen Dateien im Hauptspeicher
\item Vorteile I-Node?
\item[$\rightarrow$]Viel Weniger Speichplatzbedarf als bei FAT, Größe des Speichers proportional zu max. Anz. gleichzeitig offener Dateien, unabhängig von Plattengröße
\item[$\rightarrow$]durch geringe Größe kiann Datei lange im Hauptspeicher bleiben, auf kleine Dateien kann direkt zugegriffen werden. Die max. Größe ist ausreichend.
\item Programm, Prozess?
\item[$\rightarrow$] ein Prozess ist ein Prorgamm in Ausführung
\item Warum klappt Multitasking? Warum ist Multitasking sinnvoll?
\item[$\rightarrow$] Rechner sind so leistungsfähig dass einzelner Prozess schnell genug läuft, Viele Prozesse lasten den Prozessor nicht komplett aus, Warten auf Ein/Ausgabe
\end{itemize}

\subsection*{Kapitel 4}
\begin{itemize}
\item präemtiv?
\item[$\rightarrow$] Bs kann Prozess unterbrechen
\item Prozesszustandsdiagramm?
\item Prozesszustandsdiagramm. Unterschiede präemtiv/nicht präemtiv und mit Auslagerung/ohne?
\end{itemize}

\subsection*{Kapitel 5}
\begin{itemize}
\item Nachteile bei Software-Lösungen von wechselseitigem Ausschluss 
\item[$\rightarrow$] aktives Warten
\end{itemize}

\subsection*{Kapitel 6}
\begin{itemize}
\item Beweis?
\item[$\rightarrow$] durch Wiederspruch
\item Operationen mit Semaphoren?
\item[$\rightarrow$] up, down
\item Welche Semaphoren werden beim Prod. Kosumenten Problem benötigt
\item[$\rightarrow$] exclu, full, empty
\item Erkläre Atomare Operation, Kritische Region, wechselseitiger Ausschluss, Deadlock.
\item Vorraussetzungen für Bankier-Algorithmus
\end{itemize}

\subsection*{Kapitel 7}

\begin{itemize}
\item Welche Sheduling-Strtegien gibt es?
\item[$\rightarrow$] FCFS, RR, SJF, FCFS, SJT, HRRN, UNIX
\item Welche davon sind präemtiv?
\end{itemize}

\subsection*{Kapitel 8}

\begin{itemize}
\item Speicherhirarchie?
\item Anforderung an Speicherverwaltung
\item[$\rightarrow$] Relokation, Schutz, Gemeinsame Nutzung, Logisch Orga, physikalische Orga
\item Welche Methoden zur Speicherverwatung kennen sie?
\item[$\rightarrow$] Partitionierung,Paging, Segmentierung
\item Welche Paritionierungsvarianten kennen Sie?
\item[$\rightarrow$] Statische, Dynamische Partionierung, Buddy-Verfahren
\item Welche Zuteilungalgorithmen kennst du für DynPart?
\item[$\rightarrow$] BF, FF, NF
\item Vorteile/Nachteile?
\item Wie funktioniert Buddy?
\item Wie funktioniert einfaches Paging?
\item[$\rightarrow$] ähnlich statisches Partionieren, viele kl. Rahmen
\end{itemize}

\subsection*{Kapitel 9}

\begin{itemize}
\item Welche Arten von Sicherheit gibt es?
\end{itemize}
\end{document}