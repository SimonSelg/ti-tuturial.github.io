\documentclass{scrartcl}[11pt]
\usepackage{german}
\usepackage[utf8]{inputenc}
\usepackage[german]{babel}
\usepackage{amssymb}
\usepackage{amsmath}
\usepackage{graphicx}
\usepackage{fancyhdr}
\usepackage{lastpage}
\usepackage{braket}
\usepackage{listings}
\setlength{\parskip}{\medskipamount}
\setlength{\parindent}{1pt}


%%%%%%%%%%%%%%%%%%%%%%%%
% Kopf- und Fusszeilen %
%%%%%%%%%%%%%%%%%%%%%%%%
\pagestyle{fancy}
\lhead{
    \begin{tabular}{ll}
        Danny Stoll
    \end{tabular}
}
\chead{McKluskey}
\rhead{TI-Tutorial}
\lfoot{}
\cfoot{}
\rfoot{}

\begin{document}

%_______________________________________________________________________________
%_______________________________________________________________________________

\section*{Initialisierung}

\begin{tabular}{cc}
  \begin{tabular}{c}
    $L_0^{\{x_1, x_2, x_3, x_4\}}$ \\ \hline \hline
  \end{tabular} & $Prim_f = \emptyset$ \\
  \begin{tabular}{c}
    0000 \\ \hline
    0001 \\
    0100 \\
    1000 \\ \hline
    0101 \\
    0110 \\
    1010 \\
    1100 \\ \hline
    0111 \\
    1110 \\ \hline
    1111 \\
  \end{tabular}
\end{tabular}

\section*{1. Schleifendurchlauf}

\begin{tabular}{cccc}

  \begin{tabular}{c}
    $L_1^{\{x_1, x_2, x_3\}}$ \\ \hline \hline
    000- \\  \hline
    010- \\ \hline
    011- \\ \hline
    111- \\
  \end{tabular} &


  \begin{tabular}{c}
    $L_1^{\{x_1, x_2, x_4\}}$ \\ \hline \hline
    01-0 \\
    10-0 \\ \hline
    01-1 \\
    11-0 \\

  \end{tabular} &


  \begin{tabular}{c}
    $L_1^{\{x_1, x_3, x_4\}}$ \\ \hline \hline
    0-00 \\ \hline
    0-01 \\
    1-00 \\ \hline
    1-10 \\
  \end{tabular} &


  \begin{tabular}{c}
    $L_1^{\{x_2, x_3, x_4\}}$ \\ \hline \hline
    -000 \\ \hline
    -100 \\ \hline
    -110 \\ \hline
    -111 \\

  \end{tabular} \\ \\

  $Prim_f = \emptyset$  \\ \\

\end{tabular}


\section*{2. Schleifendurchlauf}


\begin{tabular}{ccc}

  \begin{tabular}{c}
    $L_2^{\{x_1, x_2\}}$ \\ \hline \hline
    01-- \\
  \end{tabular} &


  \begin{tabular}{c}
    $L_2^{\{x_1, x_3\}}$ \\ \hline \hline
    0-0- \\
  \end{tabular} &


  \begin{tabular}{c}
    $L_2^{\{x_1, x_4\}}$ \\ \hline \hline
    1--0 \\
  \end{tabular}  \\ \\

  \begin{tabular}{c}
    $L_2^{\{x_2, x_3\}}$ \\ \hline \hline
    -11- \\
  \end{tabular} &

  \begin{tabular}{c}
    $L_2^{\{x_2, x_4\}}$ \\ \hline \hline
    -1-0 \\
  \end{tabular} &

  \begin{tabular}{c}
    $L_2^{\{x_3, x_4\}}$ \\ \hline \hline
    --00 \\
  \end{tabular} \\ \\

  $Prim_f = \emptyset$

\end{tabular}

\pagebreak

\section*{3. Schleifendurchlauf}

\begin{tabular}{cccc}

  \begin{tabular}{c}
    $L_3^{\{x_1\}}$ \\ \hline \hline
  \end{tabular} &

  \begin{tabular}{c}
    $L_3^{\{x_2\}}$ \\ \hline \hline
  \end{tabular} &

  \begin{tabular}{c}
    $L_3^{\{x_3\}}$ \\ \hline \hline
  \end{tabular} &

  \begin{tabular}{c}
    $L_3^{\{x_4\}}$ \\ \hline \hline
  \end{tabular} \\ \\
\end{tabular}

  $Prim_f = \{  0-0-, \ 1--0, \ -11-, \ -1-0, \ --00, \ 01-- \}$ \\

\section*{Abbruch}



\begin{align*}
 &\bigcup_M L_3^M(f) = \emptyset \\
\Rightarrow & \mbox{ Abbruch der Schleife und } return \ Prim(f)
\end{align*}



\end{document}
