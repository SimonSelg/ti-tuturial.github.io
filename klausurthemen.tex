\documentclass{scrartcl}[9pt, a4paper]
\usepackage[utf8]{inputenc}
\usepackage[ngerman]{babel}

% Citations
\usepackage[numbers, sort]{natbib}
\usepackage[colorlinks, citecolor=blü, linkcolor=black, urlcolor=green, bookmarks=false, hypertexnames=trü]{hyperref}

% (Computer\item)science
\usepackage[ruled, vlined, linesnumbered]{algorithm2e}
\usepackage{amssymb, amsmath}

% Formatting
\usepackage{subcaption}
\usepackage{graphicx}
\usepackage{float}
\usepackage{fancyhdr}

\setlength\parindent{0cm}
\setlength\parskip{0.3cm}

\pagestyle{fancy}
\lhead{
\begin{tabular}{ll}
\end{tabular}
}
\chead{Klausur Themen}
\rhead{TI-Tutorat}
\lfoot{}
\cfoot{}
\rfoot{\thepage}

% ==============================================================================
% ==============================================================================
\begin{document}

\textbf{Disclaimer:} Diese Themenliste wurde von Tutoren ohne explizites Wissen über die Klausuraufgaben erstellt.

%_______________________________________________________________________________
\section*{Kodierung}

\begin{itemize}
	\item Komplement Beweise
	\item Zahlendarstellung
	      \begin{itemize}
	      	\item Betragvorzeichen
	      	\item Einer Komplement
	      	\item Zweier komplement
	      	\item Basen
	      	\item Umwandlung
	      \end{itemize}
	\item Parity Code
	\item Hamming Code
	      \begin{itemize}
	      	\item Distanz
	      	\item Erkennen, korrigieren
	      \end{itemize}
	\item Huffman Code
\end{itemize}

%_______________________________________________________________________________
\section*{Timing}

\begin{itemize}
	\item Timingdiagramme
	\item Spikefreies Umschalten / Pulsweite
	\item Kontrollogik RETI
\end{itemize}

%_______________________________________________________________________________
\section*{Verifikation}

\begin{itemize}
	\item KDNF/DNF/KNF
	\item BDDS
	      \begin{itemize}
	      	\item Umwandlung BDDs \item Schaltkreise
	      	\item Reduzierung
	      	\item Ordnung
	      \end{itemize}
\end{itemize}

%_______________________________________________________________________________
\pagebreak
\section*{Kombinatorische Logik}

\begin{itemize}
	\item Boolsche Algebra Beweise (Axiome)
	\item RS/SR/DLATCH Flipflop
	\item Arithmetische Schaltun
	\item Konstruktion
	      \begin{itemize}
	      	\item CRA
	      	\item CSA
	      	\item Multiplizierer
	      \end{itemize}
	\item PLAs
	\item Schaltkreise

	      \begin{itemize}
	      	\item Kosten/Tiefe bestimmen
	      	\item zeichnen und formalisieren
	      	\item Topologische Sortierung
	      	\item Symbolische Simulation
	      \end{itemize}

	\item KDNF/DNF/KNF
	\item Hypercubes
	\item McCluskey
	\item Minimalpolynom Beweis
	\item Primimplikanten Tafel
	      \begin{itemize}
	      	\item Petrick
	      \end{itemize}
\end{itemize}

%_______________________________________________________________________________
\pagebreak
\section*{Sequentielle Logik}

\begin{itemize}
	\item RETI Kontrollogik / Signale
	\item RETI Datenpfade
	      \begin{itemize}
	      	\item Realisierbarkeit von Befehlen
	      	\item Pfade für Befehl X
	      \end{itemize}
	\item Mealy und Mooreautomaten
	      \begin{itemize}
	      	\item Zeichnen
	      	\item Synthetisieren
	      \end{itemize}
	\item Sequentielle Synthese
	      \begin{itemize}
	      	\item Zustandskodierung
	      \end{itemize}
\end{itemize}

%_______________________________________________________________________________
\section*{RETI Assembler}

\begin{itemize}
	\item RETI Programm kommentieren/verstehen
	\item RETI Programm schreiben (unwarschl.)
\end{itemize}

%_______________________________________________________________________________
\section*{Sonstiges}

\begin{itemize}
	\item Timing in RETI (Datenpfade, ALU)
	\item Pipelining
	\item Caching
	      \begin{itemize}
	      	\item Direct Map
	      	\item Verdrängungsstrategien
	      \end{itemize}
	\item ...
\end{itemize}

\end{document}
