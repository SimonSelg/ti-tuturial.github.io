\documentclass{scrartcl}[11pt]
\usepackage{german}
\usepackage[utf8]{inputenc}
\usepackage[german]{babel}
\usepackage{amssymb}
\usepackage{amsmath}
\usepackage{graphicx}
\usepackage{fancyhdr}
\usepackage{lastpage}
\usepackage{braket}
\usepackage{listings}
\usepackage{tikz}
\usetikzlibrary{arrows,automata,positioning}

\setlength{\parskip}{\medskipamount}
\setlength{\parindent}{1pt}


%%%%%%%%%%%%%%%%%%%%%%%%
% Kopf- und Fusszeilen %
%%%%%%%%%%%%%%%%%%%%%%%%
\pagestyle{fancy}
\lhead{TI-Tutorial}
\chead{Quiz 1}
\rhead{20 Minuten}
\lfoot{}
\cfoot{}
\rfoot{}

\begin{document}

%_______________________________________________________________________________
%_______________________________________________________________________________

\section{Quine-McCluskey (10 Pkt.)}

Die Funktion $f: \mathbb{B}^4 \rightarrow \mathbb{B}$ sei durch ihre $OFF$-Menge gegeben:
$$
OFF(f) := \{0010, 0110, 1011, 1111 \}
$$

Berechnen sie alle Primimplikanten von f nach dem Verfahren von Quine-McCluskey. Geben sie
alle Zwischenschritte (d.h. die Menge $L_i^M$ und $Prim_i$) und das resultierende Minimalpolynom an.
Achten Sie darauf, zu welchem Zeitpunkt der Algorithmus die Primimplikanten erkennt.

\emph{Hinweis:} Sie dürfen in dieser Aufgabe die abkürzende Schreibweise für Monome verwenden ( z.B.
statt $\overline{x_1} \ \overline{x_2} \ \overline{x_4}: \quad 01$-$1)$.

%_______________________________________________________________________________
%_______________________________________________________________________________

\section{Kodierung (2+3+5 Pkt.)}
\begin{itemize}
  \item[a)] Geben Sie die Interpretationsfunktion $[.]_2$ für Zweierkomplementzahlen mit $n+1$ Vor- und $k$
  Nachkommastellen (also $d_n d_{n-1} \ldots d_1 d_0 d_{-1} \ldots d_{-k})$ an.
  \item[b)] Geben Sie die Werte folgender Zweierkomplementzahlen im Dezimalsystem an:
  \begin{itemize}
    \item[] 0101.10
    \item[] 1001.01
  \end{itemize}
  \item[c)] Beweisen sie folgendes Lemma: \\ \\
 \emph{Lemma}: Sei $[a]_2 = \left[a_{n-1}a_{n-2} \ldots a_0 \right]_2$ eine ganze Zahl in Zweier-Komplement-Darstellung mit n
 Vorkommastellen und keinen Nachkommastellen. Dann gilt:
 $$[\bar{a}]_2 = -[a]_2 -1 $$
 Hierbei sei $[\bar{a}]_2$ die Zahl im Zweier-Komplement, die aus $[a]_2$ durch Invertieren aller Bits
 hervorgeht. Abgesehen von der geometrischen Summenformel sollen keine Sätze aus der Vorlesung ohne
 Beweis benutzt werden. \\ \\
 \emph{Hinweis}: bei Zahlen ohne Nachkommastellen gilt k = 0, allerdings gehört k immer noch zu den
 obigen Defintionen

\end{itemize}
\end{document}
